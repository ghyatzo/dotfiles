
\usepackage[utf8]{inputenc}
\usepackage[T1]{fontenc}
\usepackage{textcomp}
\usepackage{url}

\usepackage{parskip}
\usepackage{float}
\usepackage{booktabs}
\usepackage{enumitem}
\usepackage{hyperref}
\usepackage[]{quoting}
\usepackage{emptypage}
\usepackage{subcaption}
\usepackage{multicol}
%\usepackage[dvipsnames,table,xcdraw]{xcolor}
\usepackage{amsmath, amsfonts, mathtools, amsthm, amssymb, mathtools}
\usepackage{cancel}
\usepackage{bm} % bold math
\usepackage{tikz-cd}
\usepackage{mathrsfs}
\usepackage{bm}
\usepackage{wrapfig} % allows for text wrap around figure

\newcommand{\into}{\hookrightarrow}
\newcommand{\onto}{\twoheadrightarrow}
\newcommand\N{\ensuremath{\mathbb{N}}}
\newcommand\R{\ensuremath{\mathbb{R}}}
\newcommand\Z{\ensuremath{\mathbb{Z}}}
\renewcommand\O{\ensuremath{\emptyset}}
\newcommand\Q{\ensuremath{\mathbb{Q}}}
\newcommand\C{\ensuremath{\mathbb{C}}}
\newcommand\K{\ensuremath{\mathbb{K}}}
\newcommand\F{\ensuremath{\mathbb{F}}}
\newcommand{\st}{
    \noscript\;
    \ifnum\currentgrouptype=16
        \;\middle|\;
    \else
        \;|\;
    \fi
\noscript\;}

\DeclareMathOperator{\re}{Re}
\DeclareMathOperator{\im}{Im}


% Put x\to \infty below \lim
\let\svlim\lim\def\lim{\svlim\limits}

% make implies and impledby shorter
\let\implies\Rightarrow
\let\impliedby\Leftarrow
\let\iff\Leftrightarrow
\let\epsilon\varepsilon

\usepackage{stmaryrd} % for \lightning
\newcommand\contra{\scalebox{1.5}{$\lightning$}}

\definecolor{correct}{HTML}{009900}
\newcommand\correct[2]{\ensuremath{\:}{\color{red}{#1}}\ensuremath{\to }{\color{correct}{#2}}\ensuremath{\:}}
\newcommand\green[1]{{\color{correct}{#1}}}

\newcommand\hr{
    \noindent\rule[0.5ex]{\linewidth}{0.5pt}
}

\theoremstyle{definition}
\newtheorem{definition}{Definition}[section]
\newtheorem{theorem}{Theorem}[section]
\newtheorem{proposition}{Proposition}[section]
\newtheorem{lemma}{Lemma}[section]
\newtheorem{corollary}{Corollary}[section]
\newtheorem{result}{Result}[section]
\newtheorem{example}{Example}[section]

\theoremstyle{remark}
\newtheorem*{remark}{Remark}

%% Environments
%\makeatother
%% For box around Definition, Theorem, \ldots
%\makeatother
%\usepackage{thmtools}
%\usepackage[framemethod=TikZ]{mdframed}
%\mdfsetup{skipabove=1em,skipbelow=0em}

%\theoremstyle{definition}

%\declaretheoremstyle[
%   headfont=\bfseries\sffamily\color{ForestGreen!70!black}, bodyfont=\normalfont,
%   mdframed={
%       linewidth=2pt,
%       rightline=false, topline=false, bottomline=false,
%       linecolor=ForestGreen, backgroundcolor=ForestGreen!5,
%   }
%]{thmgreenbox}


%\declaretheoremstyle[
%   headfont=\bfseries\sffamily\color{RawSienna!70!black}, bodyfont=\normalfont,
%   numbered=no,
%   mdframed={
%       linewidth=2pt,
%       rightline=false, topline=false, bottomline=false,
%       linecolor=RawSienna, backgroundcolor=RawSienna!1,
%   },
%   qed=\qedsymbol
%]{thmproofbox}

%\declaretheoremstyle[
%   headfont=\bfseries\sffamily\color{NavyBlue!70!black}, bodyfont=\normalfont,
%   mdframed={
%       linewidth=2pt,
%       rightline=false, topline=false, bottomline=false,
%       linecolor=NavyBlue, backgroundcolor=NavyBlue!5,
%   }
%]{thmbluebox}

%\declaretheoremstyle[
%   headfont=\bfseries\sffamily\color{NavyBlue!70!black}, bodyfont=\normalfont,
%   mdframed={
%       linewidth=2pt,
%       rightline=false, topline=false, bottomline=false,
%       linecolor=NavyBlue
%   }
%]{thmblueline}

%\declaretheoremstyle[
%   headfont=\bfseries\sffamily\color{RawSienna!70!black}, bodyfont=\normalfont,
%   mdframed={
%       linewidth=2pt,
%       rightline=false, topline=false, bottomline=false,
%       linecolor=RawSienna, backgroundcolor=RawSienna!5,
%   }
%]{thmredbox}

%\declaretheorem[style=thmproofbox, name=Proof]{replacementproof}
%\renewenvironment{proof}[1][\proofname]{\vspace{-10pt}\begin{replacementproof}}{\end{replacementproof}}

%\theoremstyle{thmgreenbox}
%\declaretheorem[style = thmgreenbox, name = Definitie]{definitie}
%\declaretheorem[style = thmgreenbox, name = Definition]{definition}
%%\newtheorem[nobreak=true]{definitie}{Definitie}
%\declaretheorem[style=thmredbox, name=Proposition]{proposition}
%\declaretheorem[style=thmredbox, name=Theorem]{theorem}
%\declaretheorem[style=thmredbox, name=Lemma]{lemma}
%\declaretheorem[style=thmredbox, name=Propositie]{propositie}
%\declaretheorem[style=thmredbox, name=Stelling]{stelling}
%\declaretheorem[style=thmredbox, name=Eigenschap]{eigenschap}
%\declaretheorem[style=thmredbox, name=Gevolg]{gevolg}
%\declaretheorem[style=thmredbox, name=Wet]{wet}
%\declaretheorem[style=thmredbox, name=Postulaat]{postulaat}
%\declaretheorem[style=thmredbox, numbered=no, name=Corollary]{corollary}
%\newmdtheoremenv{conclusie}{Conclusie}
%\newmdtheoremenv{toemaatje}{Toemaatje}
%\newmdtheoremenv{vermoeden}{Vermoeden}
%%\newmdtheoremenv{quote}{Quote}
%\declaretheorem[style=thmbluebox, name=Exercise, numbered=no]{exercise}
%\declaretheorem[style=thmbluebox, name=Herhaling, numbered=no]{herhaling}
%\declaretheorem[style=thmbluebox, name=Intermezzo, numbered=no]{intermezzo}
%\declaretheorem[style=thmbluebox, name=Observatie, numbered=no]{observatie}
%\declaretheorem[style=thmbluebox, name=Oefening, numbered=no]{oef}
%\declaretheorem[style=thmbluebox, name=Opmerking, numbered=no]{opmerking}
%\declaretheorem[style=thmbluebox, name=Fact, numbered=no]{fact}
%\declaretheorem[style=thmgreenbox, name=Notatie, numbered=no]{notatie}
%\declaretheorem[style=thmbluebox, name=TODO, numbered=no]{TODO}
%\newtheorem*{praktisch}{Praktisch}
%\newtheorem*{probleem}{Probleem}
%\newtheorem*{terminologie}{Terminologie}
%\newtheorem*{toepassing}{Toepassing}
%\newtheorem*{uovt}{UOVT}
%\newtheorem*{vb}{Voorbeeld}
%\newtheorem*{vraag}{Vraag}
%\newtheorem*{motivation}{Motivation}
%\newtheorem*{erratum}{Erratum}
%\newtheorem*{herinner}{Herinner}
%\newtheorem*{conjectuur}{Conjectuur}
%\newtheorem*{recall}{Recall}
%\newtheorem*{revision}{Revision}

%\newtheorem*{example}{Example}
%\newtheorem*{notation}{Notation}
%\newtheorem*{previouslyseen}{As previously seen}
%\newtheorem*{remark}{Remark}
%\newtheorem*{note}{Note}
%\newtheorem*{problem}{Problem}
%\newtheorem*{observe}{Observe}
%\newtheorem*{property}{Property}
%\newtheorem*{intuition}{Intuition}

%% End example and intermezzo environments with a small diamond (just like proof
%% environments end with a small square)
%\usepackage{etoolbox}
%\AtEndEnvironment{vb}{\null\hfill$\diamond$}%
%\AtEndEnvironment{intermezzo}{\null\hfill$\diamond$}%
%% \AtEndEnvironment{opmerking}{\null\hfill$\diamond$}%

%% Fix some spacing
%% http://tex.stackexchange.com/questions/22119/how-can-i-change-the-spacing-before-theorems-with-amsthm
%\makeatletter
%\def\thm@space@setup{%
%  \thm@preskip=\parskip \thm@postskip=0pt
%}
